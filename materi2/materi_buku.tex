\documentclass{book}
\usepackage[utf8]{inputenc}
\usepackage[indonesian]{babel}

\title{Kriptografi (Rinaldi Munir)}
\author{Disusun berdasarkan bahan kuliah}
\date{}

\begin{document}
\maketitle

\tableofcontents

\chapter{Pengantar Kriptografi}
    \section{Definisi dan Terminologi Kriptografi}
    \section{Layanan Kriptografi}
    \section{Sejarah Kriptografi}
    \section{Kriptanalisis}
    \section{Kriptografi Simetri dan Kriptografi Nirsimetri}
    \section{Fungsi Hash}
    \section{Kriptografi di Indonesia}
\chapter{Landasan Matematika}
    \section{Teori Informasi}
        \subsection{Entropy}
        \subsection{Laju Bahasa}
    \section{Teori Bilangan}
         \subsection{Sifat Pembagian Pada Bilangan Bulat}
         \subsection{Pembagi Bersama Terbesar (PBB)}
         \subsection{Algoritma Euclidean}
         \subsection{Kombinasi Lanjar}
         \subsection{Relatif Prima}
         \subsection{Aritmatika Modulo}
         \subsection{Bilangan Prima}
         \subsection{Fungsi Totient Euler}
         \subsection{Teorema Euler}
         \subsection{Akar Primitif dan Logaritma Diskrit}
    \section{Aljabar Abstrak}
         \subsection{Grup}
         \subsection{Ring}
         \subsection{Medan}
         \subsection{Medan Berhingga}
         \subsection{Medan Galois}
         \subsection{Aritmatika Polinom di dalam Medan Galois}

\chapter{Serangan Terhadap Kriptografi}
    \section{Serangan}
        \subsection{Serangan Kriptanalisis}
        \subsection{Brute-Force Attack dan Analytical Attack}
        \subsection{Passive Attack dan Active Attack}
    \section{Keamanan Algoritma Kriptografi}
    \section{Kompleksitas Serangan}

\chapter{Kriptografi Klasik}
    \section{Cipher Substitusi}
        \subsection{Caesar Cipher}
        \subsection{Kriptanalisis Terhadap Caesar Cipher}
    \section{Jenis-jenis Cipher Substitusi}
         \subsection{Cipher Abjad-Tunggal}
         \subsection{Cipher Abjad-Banyak}
         \subsection{Cipher Substitusi Homofonik}
         \subsection{Cipher Substitusi Poligram}         
    \section{Cipher Transposisi}
    \section{Super Enkripsi}
    \section{Metode Analisis Frekuensi}
    \section{Vigenere Cipher}
        \subsection{Metode Kasiski untuk menentukan panjang kunci vigenere Cipher}
        \subsection{Variasi Vigebere Cipher}
    \section{Playfair Cipher}
    \section{Affine Cipher}
    \section{Hill Cipher}
    \section{Enigma Cipher}
    \section{One-Time Pad}

\chapter{Kriptografi Modern}
    \section{Rangkaian Bit dan Operasinya}
    \section{algoritma enkripsi dengan XOR sederhana}
    \section{Kategori Chiper untuk data digital}
    \section{Cipher alir}
    \section{Pembangkit kunci alir}
    \section{Linear Feedback Shift Register (LFSR)}
    \section(Serangan Terhadap Cipher Alir)
    \section{Chiper Blok}
    \section{Electronic Code Book (ECB)}
    \section{Cipher Block Chaining (CBC)}
    \section{Cipher Feedback (CFB)}
    \section{Output Feedback (OFB)}
    \section{Counter Mode}
    \section{Prinsip-prinsip Perancangan Cipher Blok}
        \subsection{Prinsip Confusion dan Diffusion dari Shannon}
        \subsection{Cipher Berulang}
        \subsection{Jaringan Feistel}
        \subsection{Kotak-S}
\chapter{Review Beberapa Cipher Alir dan Cipher Blok}
     \section{RC4}
     \section{A5}
     \section{DES}
     \section{Double DES dan Triple DES}
     \section{GOST}
     \section{RC5}
     \section{RC6}
     \section{Advanced Encryption Standard (AES)}
     \section{Penerapan Kriptografi Simetri dalam Kehidupan Sehari-hari}
         \subsection{Transaksi dengan Mesin ATM}
         \subsection{Komunikasi dengan Telepon Seluler}
\chapter{Kriptografi Kunci Publik}         
     \section{Konsep Kriptografi Kunci Publik}
     \section{Sejarah Kriptografi Kunci Publik}
     \section{Perbandingan Kriptografi Kunci Simetri dan Kriptografi Kunci Publik}
     \section{Aplikasi Kriptografi Kunci Publik}
     \section{Algoritma RSA}
     \section{Pertukaran Kunci Diffie-Hellman}
     \section{Algoritma ElGamal}
     \section{Algoritma Knapsack}
     \section{Algoritma untuk Perpangkatan Modulo}
     \section{Pembangkitan Bilangan Prima}
     \section{Kode Program Algoritma RSA}
\chapter{Pembangkit Bilangan Acak}
    \section{Linear Congruential Generator (LCG)}
    \section{Pembangkit Bilangan Acak yang aman untuk kriptografi}
    \section{Blum Blum Shub (BBS)}
    \section{CSPRNG Berbasis Algoritma RSA}
    \section{CSPRNG Berbasis Chaos}     
\chapter{Fungsi Hash Satu Arah}
     \section{Fungsi Hash Satu Arah}
     \section{Algoritma MD5}
     \section{Secure Hash Algorithm (SHA)}
     \section{SHA-3 (Keceak)}
     \section{Message Authentication Code (MAC)}
     \section{Algoritma MAC}
\chapter{Tanda Tangan Digital}
     \section{Review Layanan Kriptografi}
     \section{Penandatanganan dengan Cara Mengenkripsi Pesan}
     \section{Tanda-tangan Digital dengan Kombinasi Fungsi Hash dan Kriptografi Kunci Publik}
     \section{Tanda-tangan Digital dengan Fungsi Hash dan Kriptografi Kunci Publik}
     \section{Digital Standard Algorithm (DSA)}
\chapter{Sertifikat Digital dan Infrastruktur Kunci Publik}
     \section{Sertifikat Digital}
     \section{X.509}
     \section{Verifikasi Sertifikat Digital}
     \section{Menggunakan Sertifikat Digital untuk Enkripsi Pesan}
     \section{Public Key Infrastructure (PKI)}
\chapter{Protokol Kriptografi}
     \section{Protokol Komunikasi dengan Sistem Kriptografi Simetri}
     \section{Protokol Komunikasi dengan Sistem Kriptografi Kunci Publik}
     \section{Protokol untuk Tanda-tangan Digital}
     \section{Protokol untuk Tanda-tangan Digital Plus Enkripsi}
     \section{Protokol Pertukaran Kunci Diffie-Hellman}
     \section{otentikasi}
     \section{Secure Socket Layer (SSL)}
\chapter{Manajemen Kunci}
     \section{Pembangkitan Kunci}
     \section{Penyebaran Kunci}
     \section{Penyimpanan Kunci}
     \section{Penggunaan Kunci}
     \section{Perubahan Kunci}
     \section{Penghancuran Kunci}
\chapter{Steganografi}
     \section{Sejarah Steganografi}
     \section{Konsep dan Terminologi Steganografi}
     \section{Kriteria Steganografi yang baik}
     \section{Ranah Penyembunyian Data}
     \section{Metode LSB}
     \section{Kombinasi Steganografi dan Kriptografi}
     \section{Steganalisis}
     \section{Watermarking}
     \section{Noiseless Steganography}

\end{document}
