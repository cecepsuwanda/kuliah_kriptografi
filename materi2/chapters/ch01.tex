\documentclass[../main.tex]{subfiles}
\begin{document}
\chapter{Pengantar Kriptografi}\label{ch:pengantar}
\section{Tujuan Pembelajaran}
Setelah mempelajari bab ini, mahasiswa mampu: (i) menjelaskan tujuan dan ruang lingkup kriptografi; (ii) mengaitkan layanan keamanan dengan primitif kriptografi; (iii) memahami konsep pembuktian keamanan tingkat tinggi; dan (iv) mengenali peran kriptografi dalam sistem nyata seperti TLS \citep{rfc8446}.

\section{Ringkasan Materi}
Kriptografi mempelajari teknik melindungi informasi dan layanan keamanan seperti kerahasiaan, integritas, autentikasi, serta nir-sangkal. Disiplin modern menekankan definisi formal dan pembuktian di bawah model ancaman, berbeda dari pendekatan klasik yang berbasis penyamaran. Aplikasi praktis menggabungkan primitif seperti cipher blok, fungsi hash, MAC, tanda tangan digital, dan protokol kunci publik \citep{bonehshoup,shannon1949}.

\section{Materi}
\subsection{Layanan Keamanan}
Empat pilar: kerahasiaan (confidentiality), integritas (integrity), autentikasi (authenticity), dan nir-sangkal (non-repudiation). Layanan ini sering digabung dalam protokol, mis. TLS 1.3 untuk komunikasi web \citep{rfc8446}.

\subsection{Primitif Dasar}
\begin{itemize}
  \item \textbf{Cipher simetris}: enkripsi efisien untuk kerahasiaan data (mis. AES). 
  \item \textbf{Fungsi hash}: ringkas pesan, deteksi perubahan, digunakan pada tanda tangan dan MAC.
  \item \textbf{MAC dan AEAD}: autentikasi pesan, sering digabung dengan enkripsi (mis. GCM).
  \item \textbf{Kripto kunci publik}: pertukaran kunci (DH), 
    tanda tangan (RSA/DSA/ECDSA).
\end{itemize}

\subsection{Model dan Pembuktian}
Tradisi modern memformalkan keamanan (mis. IND-CPA/IND-CCA) dan menggunakan permainan keamanan serta reduksi untuk membuktikan jaminan di bawah asumsi komputasional yang eksplisit \citep{bonehshoup}.

\subsection{Sejarah Singkat}
Dari sandi klasik menuju teori Shannon tentang kerahasiaan sempurna dan era kripto modern (kunci publik) \citep{shannon1949,diffiehellman1976,rsa1978}.

\section{Latihan}
\begin{enumerate}
  \item Jelaskan perbedaan antara kerahasiaan dan integritas serta contoh protokol yang memberikan keduanya.
  \item Beri contoh primitif untuk mencapai masing-masing layanan keamanan.
  \item Mengapa pembuktian keamanan diperlukan pada kriptografi modern?
\end{enumerate}

\section{Bacaan Lanjutan}
\begin{itemize}
  \item \citep{bonehshoup} untuk fondasi dan praktik modern.
  \item \citep{rfc8446} untuk studi kasus protokol produksi.
  \item \citep{shannon1949} untuk dasar-dasar teoretis.
\end{itemize}
\end{document}
