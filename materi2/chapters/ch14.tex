\documentclass[../main.tex]{subfiles}
\begin{document}
\chapter{Fungsi Hash dan Algoritma MD5}
\section{Tujuan Pembelajaran}
Memahami sifat hash kriptografis (preimage, second-preimage, collision) dan status keamanan MD5.

\section{Ringkasan Materi}
Hash kriptografis memetakan pesan ke ringkas berukuran tetap. MD5 kini tidak aman terhadap tabrakan; standar modern merekomendasikan SHA-2/SHA-3 \citep{rfc6151,fips1804,fips202}.

\section{Materi}
\subsection{Properti Keamanan}
Ketahanan preimage, second-preimage, dan collision. Peran hash pada tanda tangan, MAC (HMAC), dan integritas data.

\subsection{MD5 dan Status}
Kelemahan tabrakan praktis; jangan digunakan untuk keamanan baru \citep{rfc6151}.

\section{Latihan}
\begin{enumerate}
  \item Jelaskan mengapa tabrakan pada MD5 merusak tanda tangan digital yang mengandalkan hash.
  \item Bandingkan SHA-256 dan SHA3-256.
\end{enumerate}

\section{Bacaan Lanjutan}
\begin{itemize}
  \item \citep{rfc6151,fips1804,fips202}.
\end{itemize}
\end{document}
