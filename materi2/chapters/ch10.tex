\documentclass[../main.tex]{subfiles}
\begin{document}
\chapter{Advanced Encryption Standard (AES)}
\section{Tujuan Pembelajaran}
Memahami struktur SPN AES, ukuran kunci (128/192/256), dan properti keamanan/implementasi.

\section{Ringkasan Materi}
AES adalah standar sandi blok modern berbasis Rijndael: operasi SubBytes, ShiftRows, MixColumns, dan AddRoundKey. Aman secara luas jika dioperasikan dengan mode/parameter yang benar \citep{fips197}.

\section{Materi}
\subsection{Struktur AES}
SPN dengan S-box nonlinier, transformasi linear difusif, dan penjadwalan kunci.

\subsection{Aspek Implementasi}
Waspada pada side-channel (cache/timing); gunakan implementasi tabel-konstan atau instruksi perangkat keras (AES-NI).

\section{Latihan}
\begin{enumerate}
  \item Bandingkan AES-128 vs AES-256 dari sisi keamanan dan performa.
  \item Jelaskan hubungan AES dengan mode AEAD seperti GCM.
\end{enumerate}

\section{Bacaan Lanjutan}
\begin{itemize}
  \item \citep{fips197}.
\end{itemize}
\end{document}
