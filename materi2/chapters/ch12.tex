\documentclass[../main.tex]{subfiles}
\begin{document}
\chapter{Algoritma RSA}
\section{Tujuan Pembelajaran}
Memahami pembuatan kunci, enkripsi/dekripsi, tanda tangan RSA, padding modern, dan isu implementasi.

\section{Ringkasan Materi}
RSA berbasis faktorisasi bilangan bulat besar: \(n=pq\), \(\varphi(n)=(p-1)(q-1)\), kunci privat \(d \equiv e^{-1} \bmod \varphi(n)\). Penggunaan aman membutuhkan padding dan skema terstandar seperti RSAES-OAEP dan RSASSA-PSS \citep{pkcs1v2_2}.

\section{Materi}
\subsection{Pembuatan Kunci}
Pemilihan prima yang kuat, eksponen publik lazim \(e=65537\), proteksi CRT dan blinding.

\subsection{Enkripsi dan Tanda Tangan}
OAEP untuk enkripsi tahan CCA; PSS untuk tanda tangan tahan EUF-CMA.

\section{Latihan}
\begin{enumerate}
  \item Tunjukkan mengapa RSA tanpa padding tidak aman terhadap serangan textbook.
  \item Jelaskan peran CRT dan risiko fault attack.
\end{enumerate}

\section{Bacaan Lanjutan}
\begin{itemize}
  \item \citep{pkcs1v2_2}.
\end{itemize}
\end{document}
