\documentclass[../main.tex]{subfiles}
\begin{document}
\chapter{Otentikasi dan Tanda Tangan Digital}
\section{Tujuan Pembelajaran}
Membedakan MAC vs tanda tangan digital, memahami AEAD, dan target keamanan EUF-CMA.

\section{Ringkasan Materi}
MAC memberikan integritas dan autentikasi kunci bersama; tanda tangan memungkinkan verifikasi publik dan nir-sangkal. AEAD (mis. GCM) menggabungkan kerahasiaan dan integritas dalam satu primitive \citep{sp80038d,bonehshoup}.

\section{Materi}
\subsection{MAC}
Konstruksi HMAC, keamanan bergantung pada hash yang mendasari dan kunci rahasia.

\subsection{Tanda Tangan}
RSA-PSS, DSA, ECDSA; pemilihan parameter dan perlindungan implementasi penting.

\section{Latihan}
\begin{enumerate}
  \item Kapan sebaiknya memilih MAC vs tanda tangan digital?
  \item Jelaskan peran data terkait yang diautentikasi (AAD) pada AEAD.
\end{enumerate}

\section{Bacaan Lanjutan}
\begin{itemize}
  \item \citep{sp80038d,bonehshoup}.
\end{itemize}
\end{document}
