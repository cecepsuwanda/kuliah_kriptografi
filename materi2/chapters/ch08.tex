\documentclass[../main.tex]{subfiles}
\begin{document}
\chapter{Tipe dan Mode Algoritma Simetri}
\section{Tujuan Pembelajaran}
Memahami perbedaan sandi blok dan alir, serta mode operasi (ECB, CBC, CTR, GCM) dan sifat keamanannya.

\section{Ringkasan Materi}
Mode operasi memetakan cipher blok menjadi enkripsi pesan panjang dan, pada AEAD, menggabungkan autentikasi. Pemilihan nonce dan parameter sangat krusial untuk keamanan \citep{sp80038a,sp80038d}.

\section{Materi}
\subsection{Sandi Blok vs Sandi Alir}
Sandi blok beroperasi pada blok tetap; sandi alir menghasilkan keystream untuk di-XOR dengan plaintext.

\subsection{Mode Operasi}
\begin{itemize}
  \item ECB: tidak aman karena mengungkap pola.
  \item CBC: butuh IV acak; rentan padding oracle tanpa autentikasi.
  \item CTR: butuh nonce unik; mudah diparalelkan.
  \item GCM: AEAD dengan tag autentikasi; sangat sensitif pada reuse nonce \citep{sp80038d}.
\end{itemize}

\section{Latihan}
\begin{enumerate}
  \item Beri contoh serangan pola pada ECB.
  \item Jelaskan dampak reuse nonce pada CTR/GCM.
\end{enumerate}

\section{Bacaan Lanjutan}
\begin{itemize}
  \item \citep{sp80038a,sp80038d}.
\end{itemize}
\end{document}
