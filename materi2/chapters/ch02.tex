\documentclass[../main.tex]{subfiles}
\begin{document}
\chapter{Jenis-Jenis Serangan di dalam Kriptografi}
\section{Tujuan Pembelajaran}
Mahasiswa mampu mengklasifikasikan model serangan (COA, KPA, CPA, CCA), memahami serangan implementasi (side-channel), dan mengaitkannya dengan target keamanan standar.

\section{Ringkasan Materi}
Keamanan dinilai relatif terhadap kemampuan lawan. Desain skema harus jelas targetnya (mis. IND-CPA/IND-CCA). Implementasi menambah permukaan serangan seperti analisis waktu, daya, dan fault injection \citep{bonehshoup,bellare_rogaway_notes}.

\section{Materi}
\subsection{Model Serangan Teoretis}
\begin{itemize}
  \item Hanya ciphertext (COA)
  \item Known-plaintext (KPA)
  \item Chosen-plaintext (CPA): lawan dapat meminta enkripsi pesan pilihannya.
  \item Chosen-ciphertext (CCA): lawan memiliki orakel dekripsi terbatas.
\end{itemize}
Target umum: IND-CPA untuk kerahasiaan pasif dan IND-CCA saat dekripsi adaptif perlu dipertimbangkan.

\subsection{Serangan Implementasi}
\begin{itemize}
  \item \textbf{Timing/power analysis}: kebocoran samping dari waktu/energi.
  \item \textbf{Fault attacks}: kesalahan terinduksi mengungkap kunci.
  \item \textbf{Cache/EM leaks}: pola akses memori atau emisi elektromagnetik.
\end{itemize}
Mitigasi memerlukan disiplin implementasi dan parameter yang tepat.

\section{Latihan}
\begin{enumerate}
  \item Beri contoh skenario yang menuntut IND-CCA.
  \item Jelaskan perbedaan antara keamanan algoritme dan keamanan implementasi.
\end{enumerate}

\section{Bacaan Lanjutan}
\begin{itemize}
  \item \citep{bonehshoup} bab keamanan formal.
  \item \citep{bellare_rogaway_notes} catatan pengantar modern.
\end{itemize}
\end{document}
