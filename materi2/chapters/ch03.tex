\documentclass[../main.tex]{subfiles}
\begin{document}
\chapter{Teori Bilangan}
\section{Tujuan Pembelajaran}
Memahami aritmetika modulo, invers, teorema kecil Fermat dan Euler, orde elemen, grup siklik, dan aplikasinya pada RSA serta Diffie--Hellman.

\section{Ringkasan Materi}
Banyak skema bergantung pada struktur matematika seperti \(\mathbb{Z}_n\), grup perkalian modulo prima, dan kesulitan komputasional faktorisasi atau logaritma diskrit \citep{hac,bonehshoup}.

\section{Materi}
\subsection{Aritmetika Modulo}
Definisi kongruensi, penjumlahan, perkalian, invers modulo menggunakan Euclidean Algorithm.

\subsection{Teorema Fermat dan Euler}
Jika \(p\) prima: \(a^{p-1}\equiv 1\; (\bmod\; p)\) untuk \(a\not\equiv 0\). Umum: \(a^{\varphi(n)}\equiv 1\; (\bmod\; n)\) bila \(\gcd(a,n)=1\).

\subsection{Orde dan Grup}
Orde elemen, generator grup siklik, aplikasi pada Diffie--Hellman di grup yang cocok.

\subsection{Penerapan pada Skema}
RSA menggunakan faktorisasi \(n=pq\); keamanan DH bergantung pada sulitnya logaritma diskrit.

\section{Latihan}
\begin{enumerate}
  \item Temukan invers dari 37 modulo 101.
  \item Buktikan \(a^{\varphi(n)}\equiv 1\) untuk \(\gcd(a,n)=1\).
\end{enumerate}

\section{Bacaan Lanjutan}
\begin{itemize}
  \item \citep{hac} Bab teori bilangan.
  \item \citep{bonehshoup} ringkasan teori bilangan untuk kripto.
\end{itemize}
\end{document}
