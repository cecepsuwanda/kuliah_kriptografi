\documentclass[../main.tex]{subfiles}
\begin{document}
\chapter{Kriptografi dalam Kehidupan Sehari-hari}
\section{Tujuan Pembelajaran}
Mengaitkan konsep kripto dengan aplikasi nyata: web, perangkat seluler, penyimpanan, dan identitas.

\section{Ringkasan Materi}
Kriptografi ada di hampir semua lapisan infrastruktur TI: koneksi web aman (TLS), penyimpanan terenkripsi, tanda tangan dokumen, pembayaran digital, dan autentikasi multi-faktor \citep{rfc8446,nist_sp_800_63_3}.

\section{Materi}
\subsection{Web dan Aplikasi}
TLS 1.3, HSTS, dan kebijakan keamanan browser.

\subsection{Perangkat dan Data}
Enkripsi disk/berkas, manajemen kunci perangkat, dan backup terenkripsi.

\subsection{Identitas dan Tanda Tangan}
PKI untuk sertifikat, tanda tangan digital pada dokumen, dan standar identitas daring.

\section{Latihan}
\begin{enumerate}
  \item Uraikan alur keamanan data dari browser ke server modern.
  \item Berikan studi kasus insiden kebocoran karena manajemen kunci lemah.
\end{enumerate}

\section{Bacaan Lanjutan}
\begin{itemize}
  \item \citep{rfc8446,nist_sp_800_63_3}.
\end{itemize}
\end{document}
