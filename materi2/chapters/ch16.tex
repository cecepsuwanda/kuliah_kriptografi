\documentclass[../main.tex]{subfiles}
\begin{document}
\chapter{Digital Signature Standard (DSS): DSA dan SHA}
\section{Tujuan Pembelajaran}
Memahami DSA, parameter domain, peran hash SHA, dan implikasi keamanan.

\section{Ringkasan Materi}
DSS mengacu pada standar federal untuk tanda tangan digital, historisnya menggabungkan DSA dan keluarga SHA. Keamanan sangat bergantung pada pemilihan parameter dan keacakan nonce \(k\) yang kuat \citep{fips186_5,fips1804}.

\section{Materi}
\subsection{DSA}
Berbasis logaritma diskrit: tanda tangan \((r,s)\) memerlukan nonce \(k\) unik; kebocoran atau reuse \(k\) membocorkan kunci privat.

\subsection{Hash SHA}
SHA-2 (mis. SHA-256) dan SHA-3 untuk fungsi hash modern; MD5/SHA-1 tidak direkomendasikan untuk keamanan baru \citep{fips1804,fips202}.

\section{Latihan}
\begin{enumerate}
  \item Tunjukkan bagaimana reuse nonce \(k\) pada DSA membocorkan kunci privat.
  \item Bandingkan SHA-256 dan SHA3-256.
\end{enumerate}

\section{Bacaan Lanjutan}
\begin{itemize}
  \item \citep{fips186_5,fips1804,fips202}.
\end{itemize}
\end{document}
