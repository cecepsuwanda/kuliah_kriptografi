\documentclass[../main.tex]{subfiles}
\begin{document}
\chapter{Sistem Kriptografi Kunci Publik}
\section{Tujuan Pembelajaran}
Memahami konsep pasangan kunci, fungsi satu arah dengan pintu jebak, dan skema dasar: enkripsi, tanda tangan, pertukaran kunci.

\section{Ringkasan Materi}
Kunci publik memisahkan kunci enkripsi/verifikasi dari kunci dekripsi/penandatanganan. Keamanan tipikal bergantung pada faktorisasi, logaritma diskrit, atau varian eliptik \citep{bonehshoup,hac}.

\section{Materi}
\subsection{Konstruksi Dasar}
\begin{itemize}
  \item Enkripsi kunci publik (PKE)
  \item Tanda tangan digital
  \item Pertukaran kunci (Diffie--Hellman)
\end{itemize}

\subsection{Properti Keamanan}
IND-CPA/IND-CCA untuk PKE; EUF-CMA untuk tanda tangan. Korrektness dan efisiensi juga penting.

\section{Latihan}
\begin{enumerate}
  \item Jelaskan mengapa PKE murni jarang digunakan untuk enkripsi massal data.
  \item Bedakan EUF-CMA vs UF-CMA dan implikasinya.
\end{enumerate}

\section{Bacaan Lanjutan}
\begin{itemize}
  \item \citep{bonehshoup,hac}.
\end{itemize}
\end{document}
