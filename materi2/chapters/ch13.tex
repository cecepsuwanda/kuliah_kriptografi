\documentclass[../main.tex]{subfiles}
\begin{document}
\chapter{Algoritma Knapsack}
\section{Tujuan Pembelajaran}
Memahami kriptosistem knapsack (superincreasing) dan alasan kerentanannya.

\section{Ringkasan Materi}
Kriptosistem Merkle--Hellman berbasis subset-sum superincreasing pernah dianggap menjanjikan namun pecah melalui serangan seperti lattice basis reduction (LLL) pada variasi publik tertentu \citep{wikipedia_merkle_hellman}.

\section{Materi}
\subsection{Konstruksi Dasar}
Transformasi urutan superincreasing menjadi instansi umum melalui modular multiplication dan permutasi.

\subsection{Serangan dan Status}
Serangan LLL dan peningkatan teknik membuat varian dasar tidak aman.

\section{Latihan}
\begin{enumerate}
  \item Ilustrasikan enkripsi/dekripsi pada contoh kecil superincreasing.
  \item Jelaskan peran LLL dalam pemecahan knapsack publik.
\end{enumerate}

\section{Bacaan Lanjutan}
\begin{itemize}
  \item \citep{wikipedia_merkle_hellman}.
\end{itemize}
\end{document}
