\documentclass[../main.tex]{subfiles}
\begin{document}
\chapter{Public Key Infrastructure (PKI)}
\section{Tujuan Pembelajaran}
Memahami PKI, X.509, hierarki CA, dan validasi sertifikat.

\section{Ringkasan Materi}
PKI menyediakan pengelolaan identitas dan kunci publik dengan kepercayaan bertingkat: CA akar dan menengah, sertifikat X.509, CRL/OCSP, dan kebijakan \citep{rfc5280}.

\section{Materi}
\subsection{Sertifikat X.509}
Struktur subjek, kunci, masa berlaku, ekstensi, dan tanda tangan CA.

\subsection{Validasi}
Pemilihan jalur ke CA tepercaya, pemeriksaan masa berlaku, pencabutan (CRL/OCSP), dan penegakan kebijakan.

\section{Latihan}
\begin{enumerate}
  \item Uraikan proses validasi sertifikat TLS sisi klien.
  \item Jelaskan perbedaan CRL vs OCSP.
\end{enumerate}

\section{Bacaan Lanjutan}
\begin{itemize}
  \item \citep{rfc5280}.
\end{itemize}
\end{document}
