\documentclass[../main.tex]{subfiles}
\begin{document}
\chapter{Cipher yang Tidak Dapat Dipecahkan (Unbreakable Cipher)}
\section{Tujuan Pembelajaran}
Memahami konsep perfect secrecy dan syarat tercapainya pada One-Time Pad (OTP).

\section{Ringkasan Materi}
OTP mencapai kerahasiaan sempurna ketika kunci acak murni, sepanjang pesan, dan tidak pernah digunakan ulang. Keterbatasan logistik kunci membuat OTP jarang praktis \citep{shannon1949,wikipedia_otp}.

\section{Materi}
\subsection{Definisi Perfect Secrecy}
Distribusi ciphertext independen dari plaintext; tidak ada informasi bocor \citep{shannon1949}.

\subsection{Konstruksi OTP}
Ciphertext \(C = M \oplus K\) dengan kunci \(K\) sepanjang \(M\). Pemakaian ulang kunci membocorkan \(M_1\oplus M_2\).

\section{Latihan}
\begin{enumerate}
  \item Tunjukkan bahwa OTP memiliki perfect secrecy di bawah asumsi kunci.
  \item Demontrasikan kebocoran saat kunci digunakan ulang.
\end{enumerate}

\section{Bacaan Lanjutan}
\begin{itemize}
  \item \citep{shannon1949} dan \citep{wikipedia_otp}.
\end{itemize}
\end{document}
