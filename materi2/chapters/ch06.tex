\documentclass[../main.tex]{subfiles}
\begin{document}
\chapter{Steganografi dan Watermarking}
\section{Tujuan Pembelajaran}
Membedakan kriptografi, steganografi, dan watermarking; memahami ancaman dan metrik ketahanan.

\section{Ringkasan Materi}
Steganografi menyembunyikan keberadaan pesan dalam media penampung; watermarking menanamkan penanda (robust/fingerprint) untuk proteksi hak cipta dan penelusuran. Berbeda dari kripto, tujuan utamanya bukan kerahasiaan isi saja, tetapi juga ketidakterdeteksian dan ketahanan terhadap modifikasi \citep{cox2007digitalstego,wikipedia_steganography}.

\section{Materi}
\subsection{Model dan Metrik}
Imperceptibility, capacity, robustness. Adversary dapat melakukan kompresi, cropping, atau serangan statistik.

\subsection{Teknik Dasar}
\begin{itemize}
  \item LSB embedding pada citra; rentan terhadap kompresi lossy.
  \item Domain frekuensi (DCT/DWT) untuk watermark robust \citep{cox2007digitalstego}.
\end{itemize}

\section{Latihan}
\begin{enumerate}
  \item Rancang skema watermark robust untuk JPEG dan identifikasi trade-off.
  \item Bandingkan tujuan keamanan steganografi vs kriptografi.
\end{enumerate}

\section{Bacaan Lanjutan}
\begin{itemize}
  \item \citep{cox2007digitalstego}, \citep{wikipedia_steganography}.
\end{itemize}
\end{document}
