\documentclass[../main.tex]{subfiles}
\begin{document}
\chapter{Protokol Kriptografi}
\section{Tujuan Pembelajaran}
Memahami peran protokol dalam menggabungkan primitif untuk mencapai tujuan sistemik (kerahasiaan, integritas, autentikasi, forward secrecy).

\section{Ringkasan Materi}
Protokol menetapkan urutan pesan dan aturan derivasi kunci. Contoh: TLS 1.3 menyederhanakan handshake, menerapkan forward secrecy, dan mengautentikasi server (opsional klien) \citep{rfc8446}.

\section{Materi}
\subsection{Handshake dan Kunci}
Pertukaran kunci (mis. (EC)DHE), derivasi kunci sesi, dan pemisahan kunci untuk enkripsi dan autentikasi.

\subsection{Properti Lanjutan}
Forward secrecy, session resumption, dan 0-RTT dengan implikasi keamanan tertentu.

\section{Latihan}
\begin{enumerate}
  \item Jelaskan bagaimana TLS 1.3 mencapai forward secrecy.
  \item Bandingkan autentikasi berbasis sertifikat vs PSK.
\end{enumerate}

\section{Bacaan Lanjutan}
\begin{itemize}
  \item \citep{rfc8446}.
\end{itemize}
\end{document}
