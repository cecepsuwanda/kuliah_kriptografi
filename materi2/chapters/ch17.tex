\documentclass[../main.tex]{subfiles}
\begin{document}
\chapter{Algoritma-algoritma Pendukung Kriptografi}
\section{Tujuan Pembelajaran}
Mengulas komponen pendukung: PRF, KDF, fungsi komitmen, generator bilangan acak, dan proof-of-knowledge ringkas.

\section{Ringkasan Materi}
Ekosistem kripto modern mencakup algoritme pendukung yang memastikan keamanan end-to-end, mulai dari derivasi kunci, pemastian entropi, hingga protokol bukti \citep{sp800108,sp80090a,bonehshoup}.

\section{Materi}
\subsection{KDF}
Menurunkan kunci dari material awal; contoh NIST KDF \citep{sp800108}.

\subsection{RNG/DRBG}
Generator deterministik berbasis blok/Hash untuk menyediakan keacakan \citep{sp80090a}.

\subsection{Komitmen dan PoK}
Skema komitmen yang tersembunyi dan terikat; proof-of-knowledge untuk membuktikan kepemilikan rahasia tanpa mengungkapkannya.

\section{Latihan}
\begin{enumerate}
  \item Bedakan KDF berbasis HMAC vs berbasis counter.
  \item Jelaskan dampak RNG lemah terhadap DSA/ECDSA.
\end{enumerate}

\section{Bacaan Lanjutan}
\begin{itemize}
  \item \citep{sp800108,sp80090a,bonehshoup}.
\end{itemize}
\end{document}
