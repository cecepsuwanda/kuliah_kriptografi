\documentclass[../main.tex]{subfiles}
\begin{document}
\chapter{Algoritma Kriptografi Modern}
\section{Tujuan Pembelajaran}
Memahami perbedaan paradigma modern: keamanan komputasional, bukti reduksi, dan pemodelan orakel.

\section{Ringkasan Materi}
Kriptografi modern mengandalkan asumsi matematis dan definisi formal. Primitif inti: PRG, PRF, dan permutasi acak semu sebagai landasan cipher blok; konstruksi hash modern dan keamanan kolisi/preimage; serta tanda tangan digital berbasis skema kunci publik \citep{bonehshoup,bellare_rogaway_notes}.

\section{Materi}
\subsection{Keamanan Kerahasiaan}
IND-CPA untuk enkripsi; untuk ketahanan terhadap dekripsi adaptif dibutuhkan IND-CCA, biasanya melalui mode seperti GCM atau skema hibrida dengan bukti \citep{bellare_rogaway_notes}.

\subsection{Integritas dan Autentikasi}
MAC untuk data dengan kunci simetris; tanda tangan digital untuk penandatanganan publik.

\subsection{Konstruksi}
Dari PRF ke cipher blok (Feistel/SPN), dari kompresi ke hash (Merkle–Damgård, sponge).

\section{Latihan}
\begin{enumerate}
  \item Jelaskan mengapa IND-CPA tidak cukup untuk skenario dengan orakel dekripsi.
  \item Hubungkan PRG \(\Rightarrow\) PRF \(\Rightarrow\) block cipher secara konseptual.
\end{enumerate}

\section{Bacaan Lanjutan}
\begin{itemize}
  \item \citep{bonehshoup,bellare_rogaway_notes}.
\end{itemize}
\end{document}
