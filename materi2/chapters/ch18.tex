\documentclass[../main.tex]{subfiles}
\begin{document}
\chapter{MAC dan Pseudo-Random Generator}
\section{Tujuan Pembelajaran}
Memahami peran MAC untuk integritas dan PRG/DRBG untuk keacakan; hubungan PRG-PRF.

\section{Ringkasan Materi}
MAC (mis. HMAC) memberikan autentikasi pesan berbasis kunci; PRG memperluas benih acak menjadi keystream semu untuk enkripsi alir dan konstruksi kripto lainnya \citep{sp80090a,bellare_rogaway_notes}.

\section{Materi}
\subsection{MAC}
Definisi EUF-CMA; HMAC dan sifatnya terhadap hash yang mendasari.

\subsection{PRG/DRBG}
Konstruksi standar dan persyaratan entropi; antarmuka dan uji kesehatan pada NIST DRBG \citep{sp80090a}.

\section{Latihan}
\begin{enumerate}
  \item Kapan AEAD lebih tepat daripada kombinasi enkripsi+MAC manual?
  \item Apa dampak reuse kunci pada stream cipher terhadap keamanan?
\end{enumerate}

\section{Bacaan Lanjutan}
\begin{itemize}
  \item \citep{sp80090a,bellare_rogaway_notes}.
\end{itemize}
\end{document}
