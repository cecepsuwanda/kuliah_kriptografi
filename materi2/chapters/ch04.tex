\documentclass[../main.tex]{subfiles}
\begin{document}
\chapter{Algoritma Kriptografi Klasik}
\section{Tujuan Pembelajaran}
Membedakan sandi substitusi dan transposisi; memahami kelemahan analisis frekuensi; mengenal Caesar dan Vigen\`{e}re.

\section{Ringkasan Materi}
Sandi klasik bermanfaat pedagogis tetapi tidak aman secara modern. Analisis frekuensi dan struktur bahasa memudahkan pemecahan \citep{wikipedia_caesar,wikipedia_vigenere}.

\section{Materi}
\subsection{Substitusi Monoalfabetik}
Sandi Caesar mengganti huruf dengan pergeseran tetap; ruang kunci kecil dan mudah dipecahkan.

\subsection{Substitusi Polialfabetik}
Vigen\`{e}re menggunakan kunci sebagai pola pergeseran. Masih rentan terhadap analisis Kasiski dan indeks koincidensi.

\subsection{Transposisi}
Mengubah posisi huruf tanpa mengubah simbol; pola bahasa tetap terlihat.

\section{Latihan}
\begin{enumerate}
  \item Implementasikan sandi Caesar dan serang dengan analisis frekuensi.
  \item Dekripsi pesan Vigen\`{e}re dengan metode Kasiski.
\end{enumerate}

\section{Bacaan Lanjutan}
\begin{itemize}
  \item \citep{wikipedia_caesar} dan \citep{wikipedia_vigenere}.
\end{itemize}
\end{document}
