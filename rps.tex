\documentclass[11pt,a4paper]{article}
\usepackage[margin=2.5cm]{geometry}
\usepackage[T1]{fontenc}
\usepackage[utf8]{inputenc}
\usepackage{longtable}
\usepackage{array}
\usepackage{enumitem}
\setlist[itemize]{topsep=2pt,itemsep=2pt,parsep=0pt,partopsep=0pt}

\title{Ekstrak Kolom \textquotedblleft Mg Ke-\textquotedblright{} dan \textquotedblleft Materi Pembelajaran\textquotedblright{}}
\date{}

\begin{document}
\maketitle

\noindent\textit{Isi kolom di bawah ini merupakan kutipan langsung dari sumber.}

\renewcommand\arraystretch{1.2}
\setlength{\LTpre}{10pt}
\setlength{\LTpost}{10pt}

\begin{longtable}{>{\raggedright\arraybackslash}p{0.12\textwidth} >{\raggedright\arraybackslash}p{0.82\textwidth}}
\textbf{Mg Ke-} & \textbf{Materi Pembelajaran} \\ \hline
\endfirsthead
\textbf{Mg Ke-} & \textbf{Materi Pembelajaran} \\ \hline
\endhead

1 & \begin{itemize}
\item Pengenalan konsep kriptografi secara umum
\item Sejarah kriptografi
\item Definisi kriptografi
\item Konsep kriptografi konvensional
\end{itemize} \\

2 & \begin{itemize}
\item Sistem kripto kunci public berikut keunggulan dan kelemahannya
\item Sistem kripto kunci privat/ rahasia berikut keunggulan dan kelemahannya
\item Metode tanda tangan digital beserta keunggulan dan kelemahannya
\item Fungsi hash beserta keunggulan dan kelemahannya
\item Sertifikat digitalm beserta keunggulan dan kelemahnnya
\end{itemize} \\

3 & \begin{itemize}
\item Faktor persekutuan terbesar/ greatest common divisor (FPB/GCD)
\item Pengantar ring bilangan bulat modulo n, $Z_{n}$.
\item Keterbagian dan kongruensi bilangan bulat.
\item Algoritma Euklid untuk kalkulasi GCD.
\item Algoritma extended Euklid untuk kalkulasi invers perkalian pada ring bilangan bulat modulo n.
\end{itemize} \\

4 & \begin{itemize}
\item Sistem kongkruensi linier dan teorema sisa Tiongkok (Chinese Remainder Theorem, CRT)
\item Relatif prima dan fungsi phi Euler serta sifat-sifatnya
\item Pengantar medan hingga (finite field) $Z_{p}$ (bilangan bulat modulo p, dengan p prima)
\item Kongkruensi linier modulo p (p bilangan prima)
\item Pangkat bilangan dalam $Z_{p}$ ( power of a number in modulo prime) san teorema kecil fermat (Fermat’s Little Theorem, FLT)
\end{itemize} \\

5 & \begin{itemize}
\item Akar primitif di $Z_{p}$.
\item Residu kuadratik, kongkruensi binomial, dan symbol legendre di $Z_{p}$.
\item Logaritma diskrit di $Z_{p}$.
\end{itemize} \\

6 & \begin{itemize}
\item Sandi blok dan sandi stream
\item Data Encryption Standard (DES)
\end{itemize} \\

7 & \begin{itemize}
\item DES dan beberapa varian dari DES : iterated DES dan DESX
\item Advanced Encryption Standard (AES)
\item IDEA
\end{itemize} \\

8 & \begin{itemize}
\item Left feedback shift register(LFSR)
\item Sandi vigenere
\item Sistem kripto SEAL
\item Sistem kripto RC4
\end{itemize} \\

M-9 & EVALUASI TENGAH SEMESTER \\

10 & \begin{itemize}
\item Konsep sistem kripto kunci public
\item Teorema kecil Fermat dan aplikasinya
\item Sistem kripto Rivest-Shamir-Adleman (RSA)
\end{itemize} \\

11 & \begin{itemize}
\item Konsep dan cara kerja skema tanda tangan digital
\item Skema tanda tangan digital RSA
\item Skema tanda tangan Ong-Schnorr-Shamir
\item Metode verifikasi skema tanda tangan digital dengan sistem batch
\end{itemize} \\

12 & \begin{itemize}
\item Latar belakang dan konsp dasar pendistribusian kunci
\item Metode mendistribusikan kunci rahasia
\item Metode mendistribusikan kunci public
\item Usia kunci
\item Metode pengendalian pemakaian kunci
\item Layanan pihak ketiga yang dapat dipercaya
\end{itemize} \\

13 & \begin{itemize}
\item Fungsi hash
\item Message Authentication Code (MAC)
\item unconditionally secure authentication code
\end{itemize} \\

14 & \begin{itemize}
\item Sistem KERBEROS
\item Pretty Good Privacy
\item Universal electronic payment system
\end{itemize} \\

15 & \begin{itemize}
\item Analisis sistem kripto sederhana
\end{itemize} \\

M-16 & EVALUASI AKHIR SEMESTER \\

\end{longtable}

\end{document}
