\documentclass[12pt,a4paper,oneside]{book}
\usepackage[T1]{fontenc}
\usepackage[utf8]{inputenc}
\usepackage{lmodern} % gunakan font scalable agar microtype expansion aman
\usepackage[indonesian]{babel}
\usepackage{geometry}
\geometry{margin=2.5cm}
\usepackage{amsmath,amssymb,amsthm}
\usepackage{graphicx}
\usepackage{enumitem}
\usepackage{booktabs,longtable}
\usepackage{hyperref}
\usepackage{subfiles}
\usepackage{csquotes}
\usepackage{microtype}
\usepackage{natbib}
\usepackage{xcolor}
\hypersetup{colorlinks=true,linkcolor=blue,citecolor=blue,urlcolor=blue}
\setlist{nosep}

% Penomoran lingkungan matematis per-bab
\numberwithin{equation}{chapter}
\theoremstyle{definition}
\newtheorem{definition}{Definisi}[chapter]
\newtheorem{example}{Contoh}[chapter]
\theoremstyle{plain}
\newtheorem{theorem}{Teorema}[chapter]
\newtheorem{lemma}{Lema}[chapter]

% Notasi ring/medan yang sering digunakan
\newcommand{\Z}{\mathbb{Z}}
\newcommand{\Zn}{\mathbb{Z}_n}
\newcommand{\Zp}{\mathbb{Z}_p}

\title{Buku Ajar Kriptografi\\\large Program Studi S1 Matematika\\FMIPA Universitas Bale Bandung}
\author{Tim Pengajar Kriptografi}
\date{\today}

\begin{document}
\frontmatter
\maketitle
\tableofcontents

\mainmatter
% Bab-bab (sesuai RPS)
\subfile{chapters/ch01}
\subfile{chapters/ch02}
\subfile{chapters/ch03}
\subfile{chapters/ch04}
\subfile{chapters/ch05}
\subfile{chapters/ch06}
\subfile{chapters/ch07}
\subfile{chapters/ch08}
\subfile{chapters/ch09}
\subfile{chapters/ch10}
\subfile{chapters/ch11}
\subfile{chapters/ch12}
\subfile{chapters/ch13}
\subfile{chapters/ch14}

\backmatter
\bibliographystyle{plainnat}
\bibliography{references}
\end{document}
