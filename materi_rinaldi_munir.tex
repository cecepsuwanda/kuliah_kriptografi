\documentclass{book}
\usepackage[utf8]{inputenc}
\usepackage[indonesian]{babel}

\title{Kriptografi (Rinaldi Munir)}
\author{Disusun berdasarkan bahan kuliah}
\date{}

\begin{document}
\maketitle

\tableofcontents

\chapter{Pengantar Kriptografi}
\chapter{Jenis-Jenis Serangan di dalam Kriptografi}
\chapter{Teori Bilangan}
\chapter{Algoritma Kriptografi Klasik}
\chapter{Cipher yang Tidak Dapat Dipecahkan (Unbreakable Cipher)}
\chapter{Steganografi dan Watermarking}
\chapter{Algoritma Kriptografi Modern}
\chapter{Tipe dan Mode Algoritma Simetri}
\chapter{Data Encryption Standard (DES)}
\chapter{Advanced Encryption Standard (AES)}
\chapter{Sistem Kriptografi Kunci Publik}
\chapter{Algoritma RSA}
\chapter{Algoritma Knapsack}
\chapter{Fungsi Hash dan Algoritma MD5}
\chapter{Otentikasi dan Tanda Tangan Digital}
\chapter{Digital Signature Standard (DSS): DSA dan SHA}
\chapter{Algoritma-algoritma Pendukung Kriptografi}
\chapter{MAC dan Pseudo-Random Generator}
\chapter{Protokol Kriptografi}
\chapter{Public Key Infrastructure (PKI)}
\chapter{Manajemen Kunci}
\chapter{Kriptografi dalam Kehidupan Sehari-hari}

\end{document}
