\documentclass[../main.tex]{subfiles}
\begin{document}
\chapter{LFSR, Vigen\`{e}re, SEAL, RC4}

\section{Tujuan Pembelajaran}
Mahasiswa memahami LFSR, sandi Vigen\`{e}re, dan sandi alir modern SEAL dan RC4 beserta aspek keamanannya.

\section{Linear Feedback Shift Register (LFSR)}
LFSR menghasilkan deret biner melalui umpan balik linear atas \(\mathbb{F}_2\). Banyak dipakai sebagai komponen pembangkit keystream; tetapi lineritas memudahkan kriptoanalisis jika digunakan tanpa pengacakan tambahan \citep{menezes}.

Untuk mengatasi linearitas, skema modern menggabungkan beberapa LFSR dengan fungsi nonlinier atau menggunakan konstruk modern hasil proyek eSTREAM.

\section{Sandi Vigen\`{e}re}
Sandi klasik polialfabetik menggunakan kunci kata untuk menentukan pergeseran siklik per posisi. Rentan terhadap analisis Kasiski dan indeks koincidensi; tidak aman secara modern \citep{stallings}.

\section{SEAL}
SEAL adalah sandi alir berkecepatan tinggi berbasis fungsi pseudo-acak yang dipetakan dari kunci dan nonce, dirancang untuk efisiensi perangkat lunak \citep{seal}.

\section{RC4}
RC4 adalah sandi alir terkenal dengan inisialisasi KSA dan PRGA yang menghasilkan keystream. Ditemukan berbagai bias pada keluaran awal sehingga butuh pembuangan byte awal; banyak standar modern tidak lagi merekomendasikan RC4 \citep{rc4,stallings}.

Alternatif modern yang disarankan: ChaCha20-Poly1305, distandardkan untuk protokol Internet \citep{rfc8439}. Portofolio eSTREAM juga merekomendasikan beberapa kandidat untuk perangkat lunak dan perangkat keras \citep{estream}.

\section{Latihan}
\begin{enumerate}
  \item Jelaskan mengapa LFSR murni tidak cocok sebagai sandi alir modern.
  \item Demonstrasikan enkripsi dan dekripsi singkat dengan sandi Vigen\`{e}re.
  \item Mengapa RC4 tidak direkomendasikan pada protokol modern seperti TLS?
  \item Implementasikan ChaCha20 secara ringkas dan ujikan melawan test vector standar (rujuk \citep{rfc8439}).
\end{enumerate}

\end{document}
