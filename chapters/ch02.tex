\documentclass[../main.tex]{subfiles}
\begin{document}
\chapter{Lanskap Primitif Modern: Kunci Publik, Tanda Tangan, Hash, Sertifikat}

\section{Tujuan Pembelajaran}
Mahasiswa mampu membedakan sistem kunci publik dan privat (simetris), menjelaskan fungsi hash, tanda tangan digital, serta konsep sertifikat digital dan infrastruktur kunci publik (PKI).

\section{Sistem Kunci Publik vs. Kunci Privat}
\textbf{Kunci privat/simetris}: satu kunci rahasia bersama. Contoh penggunaan: AES dalam mode operasi yang aman. Keunggulan: sangat cepat; kelemahan: distribusi kunci.

\textbf{Kunci publik/asimetris}: pasangan kunci (publik, privat). Publik untuk enkripsi atau verifikasi, privat untuk dekripsi atau penandatanganan. Memudahkan distribusi kunci, tetapi relatif lebih lambat \citep{katzlindell,stallings}.

\section{Fungsi Hash Kriptografis}
\begin{definition}[Fungsi Hash]
Pemetaan deterministik dari string panjang ke nilai tetap (digest) dengan sifat tahan tabrakan, tahan pra-citra, dan tahan pra-citra kedua (secara ideal).
\end{definition}
Penggunaan: ringkas pesan, membangun MAC dan tanda tangan, verifikasi integritas \citep{menezes}.

\section{Tanda Tangan Digital}
Skema tanda tangan menjamin keaslian dan nir-sangkal. Pihak penandatangan menggunakan kunci privat untuk menghasilkan tanda tangan atas pesan (sering atas hash pesan); pihak verifikator menggunakan kunci publik untuk memverifikasi \citep{katzlindell}.

\section{Sertifikat Digital dan PKI}
Sertifikat mengikat identitas dengan kunci publik menggunakan tanda tangan dari \emph{Certificate Authority} (CA). PKI menyediakan prosedur penerbitan, pencabutan, dan validasi sertifikat. Keamanannya bergantung pada kebijakan dan keandalan CA \citep{stallings}.

\section{Latihan}
\begin{enumerate}
  \item Jelaskan mengapa fungsi hash harus tahan tabrakan. Apa dampaknya pada tanda tangan digital jika tidak?
  \item Bandingkan model distribusi kunci pada sistem simetris vs. asimetris.
  \item Berikan contoh penggunaan sertifikat digital pada protokol TLS.
\end{enumerate}

\end{document}
