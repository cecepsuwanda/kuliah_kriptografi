\documentclass[../main.tex]{subfiles}
\begin{document}
\chapter{DES dan Variannya; AES; IDEA}

\section{Tujuan Pembelajaran}
Mahasiswa memahami varian DES (iterated DES, DESX), prinsip AES, dan gambaran IDEA.

\section{Varian DES}
\textbf{Triple-DES} mengaplikasikan DES tiga kali (mis. EDE) untuk memperpanjang keamanan. \textbf{DESX} menambahkan whitening kunci untuk memperluas ruang kunci efektif. Meskipun lebih aman dari DES tunggal, performanya kalah dari AES \citep{stallings}.

\section{Advanced Encryption Standard (AES)}
AES (Rijndael) adalah sandi blok 128-bit dengan ukuran kunci 128/192/256-bit. Bukan Feistel, melainkan transformasi putaran: SubBytes, ShiftRows, MixColumns, AddRoundKey yang didefinisikan atas medan hingga \(\mathbb{F}_{2^8}\). AES adalah standar de facto saat ini \citep{nist_aes}.

\section{IDEA}
International Data Encryption Algorithm menggunakan operasi campuran penjumlahan modulo \(2^{16}\), XOR, dan perkalian modulo \(2^{16}+1\). Desain ini menyeimbangkan difusi dan konfusi dengan operasi heterogen \citep{stallings}.

\section{Latihan}
\begin{enumerate}
  \item Bandingkan keamanan relatif DES, 3DES, dan AES.
  \item Mengapa AES menggunakan \(\mathbb{F}_{2^8}\) pada operasi S-box?
  \item Beri contoh peran whitening pada DESX.
\end{enumerate}

\end{document}
