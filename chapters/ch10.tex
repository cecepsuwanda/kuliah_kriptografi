\documentclass[../main.tex]{subfiles}
\begin{document}
\chapter{Tanda Tangan Digital: Konsep, RSA, Ong--Schnorr--Shamir}

\section{Tujuan Pembelajaran}
Mahasiswa memahami tujuan dan cara kerja tanda tangan digital, skema RSA, serta gambaran skema Ong--Schnorr--Shamir.

\section{Konsep dan Model Keamanan}
Tanda tangan digital menyediakan otentikasi, integritas, dan nir-sangkal. Keamanan formal: unforgeability under chosen-message attack (UF-CMA). Praktik: menandatangani \emph{hash} pesan, bukan pesan mentah \citep{katzlindell}.

Dalam praktik, parameter dan padding mempengaruhi keamanan. Model UF-CMA menjadi acuan evaluasi skema standar modern.

\section{Tanda Tangan RSA}
Kunci seperti RSA enkripsi. Tanda tangan atas pesan \(m\) biasanya pada digest \(H(m)\): \(\sigma\equiv H(m)^d\bmod n\). Verifikasi: cek \(H(m)\equiv \sigma^e\bmod n\). Gunakan padding dan skema standar seperti RSA-PSS; spesifikasi dan parameter terdapat pada PKCS\#1 \citep{katzlindell,stallings,rfc8017}.

\paragraph{EdDSA/ECDSA/DSA Singkat.} Skema berbasis eliptik (ECDSA, EdDSA) menawarkan ukuran tanda tangan kecil dan performa baik. Penggunaan \emph{nonce} deterministik mencegah kebocoran kunci privat saat entropi lemah \citep{rfc6979,rfc8032,fips186-5}.

\section{Ong--Schnorr--Shamir (OSS)}
OSS adalah keluarga skema tanda tangan awal berbasis gagasan trapdoor, menawarkan efisiensi tertentu namun jarang digunakan dibanding ECDSA/RSA modern. Fokus pembelajaran: struktur umum tanda tangan dan verifikasi batch \citep{stallings}.

\section{Verifikasi Batch}
Untuk kelas skema tertentu (mis. tanda tangan yang bersifat homomorfik), beberapa tanda tangan dapat diverifikasi sekaligus guna efisiensi. Perlu analisis keamanan untuk mencegah serangan yang mengeksploitasi penggabungan \citep{katzlindell}.

\section{Latihan}
\begin{enumerate}
  \item Uraikan perbedaan RSA-PKCS\#1 v1.5, RSA-PSS, dan implikasi keamanannya.
  \item Mengapa tanda tangan dilakukan atas hash, bukan pesan asli?
  \item Berikan sketsa bagaimana verifikasi batch dapat menghemat waktu untuk tanda tangan bergaya RSA.
  \item Tunjukkan bagaimana kegagalan \emph{nonce} acak pada ECDSA dapat membocorkan kunci privat; jelaskan mitigasi deterministik \citep{rfc6979}.
\end{enumerate}

\end{document}
