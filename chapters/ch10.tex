\documentclass[../main.tex]{subfiles}
\begin{document}
\chapter{Tanda Tangan Digital: Konsep, RSA, Ong--Schnorr--Shamir}

\section{Tujuan Pembelajaran}
Mahasiswa memahami tujuan dan cara kerja tanda tangan digital, skema RSA, serta gambaran skema Ong--Schnorr--Shamir.

\section{Konsep dan Model Keamanan}
Tanda tangan digital menyediakan otentikasi, integritas, dan nir-sangkal. Keamanan formal: unforgeability under chosen-message attack (UF-CMA). Praktik: menandatangani \emph{hash} pesan, bukan pesan mentah \citep{katzlindell}.

\section{Tanda Tangan RSA}
Kunci seperti RSA enkripsi. Tanda tangan atas pesan \(m\) biasanya pada digest \(H(m)\): \(\sigma\equiv H(m)^d\bmod n\). Verifikasi: cek \(H(m)\equiv \sigma^e\bmod n\). Gunakan padding dan skema standar seperti RSA-PSS \citep{katzlindell,stallings}.

\section{Ong--Schnorr--Shamir (OSS)}
OSS adalah keluarga skema tanda tangan awal berbasis gagasan trapdoor, menawarkan efisiensi tertentu namun jarang digunakan dibanding ECDSA/RSA modern. Fokus pembelajaran: struktur umum tanda tangan dan verifikasi batch \citep{stallings}.

\section{Verifikasi Batch}
Untuk kelas skema tertentu (mis. tanda tangan yang bersifat homomorfik), beberapa tanda tangan dapat diverifikasi sekaligus guna efisiensi. Perlu analisis keamanan untuk mencegah serangan yang mengeksploitasi penggabungan \citep{katzlindell}.

\section{Latihan}
\begin{enumerate}
  \item Uraikan perbedaan RSA-PKCS\#1 v1.5, RSA-PSS, dan implikasi keamanannya.
  \item Mengapa tanda tangan dilakukan atas hash, bukan pesan asli?
  \item Berikan sketsa bagaimana verifikasi batch dapat menghemat waktu untuk tanda tangan bergaya RSA.
\end{enumerate}

\end{document}
