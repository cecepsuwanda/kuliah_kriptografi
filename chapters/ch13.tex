\documentclass[../main.tex]{subfiles}
\begin{document}
\chapter{Kerberos, PGP, dan Sistem Pembayaran Elektronik}

\section{Tujuan Pembelajaran}
Mahasiswa memahami peran Kerberos, PGP, dan gambaran sistem pembayaran elektronik universal.

\section{Kerberos}
Kerberos adalah protokol otentikasi jaringan berbasis tiket yang mengandalkan KDC (AS dan TGS). Prinsip: otentikasi mutual, tiket sementara (TGT), dan sesi kunci sementara untuk layanan \citep{kerberos,stallings}.

\section{Pretty Good Privacy (PGP)}
PGP menggabungkan kriptografi simetris dan asimetris: enkripsi pesan dengan kunci sesi simetris (mis. AES), lalu kunci sesi dienkripsi dengan kunci publik penerima; juga menyediakan tanda tangan dan manajemen kunci ala \emph{web of trust} \citep{pgp,stallings}.

\section{Sistem Pembayaran Elektronik}
Gagasan sistem pembayaran elektronik universal mencakup aspek privasi, otentikasi, integritas, dan non-repudiation. Contoh komponen: token, tanda tangan, bukti tanpa pengungkapan penuh, serta infrastruktur sertifikat. Rancang bangun membutuhkan regulasi dan tata kelola tambahan.

\section{Latihan}
\begin{enumerate}
  \item Jelaskan alur tiket Kerberos mulai dari AS hingga layanan akhir.
  \item Uraikan bagaimana PGP menggabungkan kunci simetris dan asimetris.
  \item Identifikasi tantangan keamanan dan regulasi pada sistem pembayaran elektronik berskala besar.
\end{enumerate}

\end{document}
