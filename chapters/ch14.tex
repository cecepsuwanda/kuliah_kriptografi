\documentclass[../main.tex]{subfiles}
\begin{document}
\chapter{Analisis Sistem Kripto Sederhana}

\section{Tujuan Pembelajaran}
Mahasiswa mampu melakukan analisis dasar terhadap sistem kripto sederhana dan mengidentifikasi kelemahannya.

\section{Metodologi Analisis}
Langkah-langkah umum: model ancaman, asumsi penyerang, enumerasi ruang kunci, analisis statistik (mis. frekuensi), dan serangan yang memanfaatkan struktur (linearitas, periode pendek, bias) \citep{stallings}. Untuk sandi blok modern, dua pendekatan utama adalah kriptoanalisis linear dan diferensial \citep{heys_tutorial}.

\section{Contoh: Analisis Sandi Substitusi Tunggal}
Sandi substitusi tunggal mempertahankan distribusi frekuensi huruf. Serangan analisis frekuensi, bigram/trigram, dan heuristik bahasa dapat memulihkan pemetaan kunci secara bertahap.

\section{Contoh: Analisis RC4 Awal}
Bias pada byte awal keystream RC4 memungkinkan kebocoran informasi kunci pada protokol tertentu jika \emph{nonce} berulang, sehingga mitigasi dengan pembuangan byte awal tidak cukup pada banyak skenario modern \citep{rc4}.

\section{Catatan Etika dan Tanggung Jawab}
Analisis kriptografi harus dilakukan pada lingkungan terkendali dan sistem yang Anda miliki izin untuk menguji. Tujuannya adalah meningkatkan desain dan implementasi, bukan mengeksploitasi kelemahan.

\section{Latihan}
\begin{enumerate}
  \item Lakukan analisis frekuensi pada pesan terenkripsi dengan substitusi tunggal untuk menebak sebagian kunci.
  \item Identifikasi asumsi penyerang pada analisis yang Anda lakukan.
  \item Usulkan perbaikan desain untuk menutup celah yang ditemukan.
  \item (Opsional) Lakukan studi literatur ringkas tentang kriptoanalisis linear dan diferensial pada AES-Reduced-Round dengan merujuk \citep{heys_tutorial}.
\end{enumerate}

\end{document}
