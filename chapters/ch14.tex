\documentclass[../main.tex]{subfiles}
\begin{document}
\chapter{Analisis Sistem Kripto Sederhana}

\section{Tujuan Pembelajaran}
Mahasiswa mampu melakukan analisis dasar terhadap sistem kripto sederhana dan mengidentifikasi kelemahannya.

\section{Metodologi Analisis}
Langkah-langkah umum: model ancaman, asumsi penyerang, enumerasi ruang kunci, analisis statistik (mis. frekuensi), dan serangan yang memanfaatkan struktur (linearitas, periode pendek, bias) \citep{stallings}.

\section{Contoh: Analisis Sandi Substitusi Tunggal}
Sandi substitusi tunggal mempertahankan distribusi frekuensi huruf. Serangan analisis frekuensi, bigram/trigram, dan heuristik bahasa dapat memulihkan pemetaan kunci secara bertahap.

\section{Contoh: Analisis RC4 Awal}
Bias pada byte awal keystream RC4 memungkinkan kebocoran informasi kunci pada protokol tertentu jika \emph{nonce} berulang, sehingga mitigasi dengan pembuangan byte awal tidak cukup pada banyak skenario modern \citep{rc4}.

\section{Latihan}
\begin{enumerate}
  \item Lakukan analisis frekuensi pada pesan terenkripsi dengan substitusi tunggal untuk menebak sebagian kunci.
  \item Identifikasi asumsi penyerang pada analisis yang Anda lakukan.
  \item Usulkan perbaikan desain untuk menutup celah yang ditemukan.
\end{enumerate}

\end{document}
