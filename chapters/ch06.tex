\documentclass[../main.tex]{subfiles}
\begin{document}
\chapter{Sandi Blok dan Sandi Alir; DES}

\section{Tujuan Pembelajaran}
Mahasiswa memahami perbedaan sandi blok dan sandi alir, struktur tinggi DES, dan isu keamanannya.

\section{Sandi Blok vs. Sandi Alir}
Sandi blok memproses blok tetap (mis. 64/128 bit) dengan kunci rahasia, memerlukan mode operasi (ECB, CBC, CTR, GCM). Sandi alir menghasilkan keystream untuk di-XOR dengan plaintext. Pemilihan mode sangat krusial bagi keamanan \citep{stallings,menezes}.

\section{Data Encryption Standard (DES)}
DES adalah sandi blok 64-bit dengan kunci efektif 56-bit, menggunakan struktur Feistel 16 putaran. Kini DES dianggap tidak aman karena ruang kunci kecil (serangan brute force) dan kelemahan desain historis \citep{nist_des,stallings}.

\paragraph{Ringkas Struktur.} DES menggunakan permutasi awal/akhir, ekspansi, S-box nonlinier, dan P-box permutasi. Desain Feistel memastikan dekripsi menggunakan algoritma yang sama dengan urutan subkunci terbalik.

\section{Isu Keamanan dan Penggantinya}
Serangan pencarian kunci menyeluruh dan teknik kriptoanalisis modern melemahkan DES. Triple-DES dan akhirnya AES menggantikan DES pada standar internasional \citep{nist_des,nist_aes}.

\section{Latihan}
\begin{enumerate}
  \item Jelaskan perbedaan konseptual antara sandi blok dan sandi alir.
  \item Mengapa kunci 56-bit dianggap tidak memadai saat ini?
  \item Sebutkan keuntungan struktur Feistel untuk desain sandi blok.
\end{enumerate}

\end{document}
