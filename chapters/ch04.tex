\documentclass[../main.tex]{subfiles}
\begin{document}
\chapter{Sistem Kongruensi Linier, CRT, dan Teorema Kecil Fermat}

\section{Tujuan Pembelajaran}
Mahasiswa menguasai penyelesaian kongruensi linier, Teorema Sisa Tiongkok (CRT), fungsi phi Euler, medan hingga \(\Zp\), dan Teorema Kecil Fermat beserta aplikasinya.

\section{Kongruensi Linier dan Fungsi \(\varphi\) Euler}
Persamaan \(ax\equiv b\pmod n\) memiliki solusi jika dan hanya jika \(\gcd(a,n)\mid b\). Jumlah kelas sisa invertibel di \(\Zn\) adalah \(\varphi(n)\). Untuk \(n=\prod p_i^{e_i}\), \(\varphi(n)=\prod p_i^{e_i-1}(p_i-1)\) \citep{hoffstein}. Fungsi \(\varphi\) bersifat multiplikatif dan sangat penting pada teori bilangan komputasional serta kriptografi (mis. RSA). Bacaan terbuka: \citep{shoup_nt}.

\section{Teorema Sisa Tiongkok (CRT)}
Jika \(n_1,\dots,n_k\) saling bebas, maka sistem
\[ x\equiv a_i \pmod{n_i},\quad i=1,\dots,k \]
mempunyai solusi unik modulo \(N=\prod n_i\). Salah satu konstruksi eksplisit: \(x=\sum a_i M_i y_i\) dengan \(M_i=N/n_i\) dan \(y_i\equiv M_i^{-1}\pmod{n_i}\). CRT penting untuk percepatan perhitungan modular (misal implementasi RSA) \citep{menezes}.

\paragraph{Contoh.} Selesaikan \(x\equiv 2\pmod 3\), \(x\equiv 3\pmod 5\), \(x\equiv 2\pmod 7\). Diperoleh \(N=105\), \(M_1=35\), \(y_1\equiv 2\), \(M_2=21\), \(y_2\equiv 1\), \(M_3=15\), \(y_3\equiv 1\). Maka \(x\equiv 2\cdot35\cdot2+3\cdot21\cdot1+2\cdot15\cdot1\equiv 23\pmod{105}\).

\section{Medan Hingga \(\Zp\) dan FLT}
Untuk prima \(p\), \(\Zp\) adalah medan: setiap elemen non-nol memiliki invers.
\begin{theorem}[Teorema Kecil Fermat]
Jika \(p\) prima dan \(a\not\equiv 0\pmod p\), maka \(a^{p-1}\equiv 1\pmod p\). Secara umum, \(a^p\equiv a\pmod p\) untuk semua \(a\in\Z\) \citep{rosen}.
\end{theorem}
Aplikasi: menguji invertibilitas, menyederhanakan perpangkatan modular, dan fondasi RSA \citep{rsa}.

\paragraph{Contoh.} Selesaikan \(7x\equiv 1\pmod{19}\). Karena \(7^{18}\equiv 1\), maka \(7^{-1}\equiv 7^{17} \bmod 19\). Perhitungan cepat (eksponeniasi biner/square-and-multiply) memberi \(7^{-1}\equiv 11\bmod 19\).

\section{Eksponeniasi Modular Efisien}
Algoritma \emph{square-and-multiply} menghitung \(a^e\bmod n\) dalam \(O(\log e)\) langkah perkalian modular. Teknik ini esensial dalam RSA dan Diffie--Hellman.

\section{Latihan}
\begin{enumerate}
  \item Selesaikan sistem: \(x\equiv 2\pmod 3\), \(x\equiv 3\pmod 5\), \(x\equiv 2\pmod 7\).
  \item Hitung \(\varphi(2^4\cdot 3^2\cdot 5)\).
  \item Buktikan atau beri argumen mengapa \(a^{\varphi(n)}\equiv 1\pmod n\) jika \(\gcd(a,n)=1\) (Teorema Euler).
  \item Implementasikan eksponeniasi modular cepat dan gunakan untuk menghitung \(3^{12345}\bmod 1001\).
  \item Tunjukkan bagaimana CRT mempercepat dekripsi RSA dengan memecah modulo \(n=pq\) menjadi modulo \(p\) dan \(q\).
\end{enumerate}

\end{document}
